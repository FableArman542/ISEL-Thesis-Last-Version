%!TEX root = ../template.tex
%%%%%%%%%%%%%%%%%%%%%%%%%%%%%%%%%%%%%%%%%%%%%%%%%%%%%%%%%%%%%%%%%%%%
%% Config/_files.tex
%% ISEL thesis configuration file
%%%%%%%%%%%%%%%%%%%%%%%%%%%%%%%%%%%%%%%%%%%%%%%%%%%%%%%%%%%%%%%%%%%%

\typeout{NT FILE Config/_files.tex}%

%=============================================================================
% THE FILES
%=============================================================================

%------------------------------------------------------------
% Except for the bibliography, all the other FILE NAMES below
% inside braces must correspond to a file with extension
% ".tex" and located in the "Chapters" folder
%------------------------------------------------------------

%%------------------------------------------------------------
%% All the names inside braces below sould correspond to a file
%%     with extension ".tex" and located in the "Chapters" folder
%%------------------------------------------------------------


%%------------------------------------------------------------
% Statement text. Will only be considered for final documents, i.e., "bsc" and "msc", otherwise, it will be silently ignored
\statementfile{statement}

%%------------------------------------------------------------
% File with the dedicatory text. Will only be considered for final documents,
% i.e., "bsc" and "msc", otherwise, it will be silently ignored
% syntax: \dedicatoryfile{filename}
\dedicatoryfile{dedicatory}

%%------------------------------------------------------------
% File with the acknowledgments text. Will only be considered for final documents,
% i.e., "bsc" and "msc", otherwise, it will be silently ignored
% syntax: \acknowledgementsfile{filename}
\acknowledgementsfile{acknowledgements}

%%------------------------------------------------------------
% File with the quote text. Will only be considered for final documents,
% i.e., "bsc" and "msc", otherwise, it will be silently ignored
% syntax: \quotefile{filename}
% \quotefile{quote}	

%%------------------------------------------------------------
% Abstract files in multiple languages
%   syntax: \abstractfile[language]{filename}
\abstractfile[pt]{abstract-pt}	% Abstract in Portuguese
\abstractfile[en]{abstract-en}	% Abstract in English

%%------------------------------------------------------------
% Lists of Glossary, Acronyms and Symbols
% syntax: \glossaryfile{filename} & \acronymsfile{filename}
\glossaryfile{glossary}
\setglossarystyle{super}
\acronymsfile{acronyms}

%%------------------------------------------------------------
% BibTeX bibliography files and customization.
% May be used multiple times with a single file nae each time.
% syntax: \addbibfile{filename.bib}
\addbibfile{bibliography}
%\addbibfile{anotherbibfile.bib}

%%------------------------------------------------------------
% Main text - document chapters
%   syntax: \addfile{filename}
\addfile{chapter1}
\addfile{chapter2}
\addfile{chapter3}
\addfile{chapter4}
\addfile{chapter5}
%\addfile{chapter-last}

%%------------------------------------------------------------
% Text appendices
% syntax: \appendixfile{filename}
% \appendixfile{appendix1}
% \appendixfile{appendix2}

%%------------------------------------------------------------
% Text annexes
% syntax: \annexfile{filename}
% \annexfile{annex1}
% \annexfile{annex2}

%%------------------------------------------------------------
% Path to put code scripts files
\lstinputpath{Chapters/scripts}

%%------------------------------------------------------------
% The Table of Contents is always printed.
% The other lists below may be commented and omitted.
\addlisttofrontmatter{\listoffigures}	    % The List of Figures. Comment to omit.
\addlisttofrontmatter{\listoftables} 	    % The List of Tables. Comment to omit.
\addlisttofrontmatter{\lstlistoflistings}   % The List of Code Listings. Comment to omit.
\addlisttofrontmatter{\glsnogroupskiptrue\setlength{\glsdescwidth}{0.8\textwidth}\printnoidxglossaries}	    % The list of glosssaries and acronyms

