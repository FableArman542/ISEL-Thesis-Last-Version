\chapter{Conclusions}
\label{cha:conclusions}

The following chapter pretends to summarize the problem, and the proposed solution, providing a clear overview of the path followed to reach the end project that is showcased in this thesis.
The results from the testing such as feedback will be sumed and some observations will be made regarding the overall project.

In section \ref{futureWork}, the future work will be strategized so that the project has more impact over time.

% \section{Summary of the work}
    % - [ ] resumo do problema
	% - [ ] resumo da solução proposta 
	% - [ ] resumo dos resultados obtidos
	% - [ ] resumo das principais conclusões do trabalho

The goal of this thesis was to develop mini-games having in consideration children with learning disabilities. Since the focus group of this project has several distinct disabilities, an analysis was conducted of every single one of them, in order to understand the intersections of the characteristics of these issues, or their differences. The essence of this issue is the segregation of children when it comes experiencing ways of learning. The goal is to give tools to these children, so that they can overcome their challenges, and feel a closer experience of learning as their peers.

To reach an optiomal solution there were several different areas to tackle. After the initial research, the project was separated into multiple parts: the technical architecture, design system and, testing and analyzing.

Before the technical implementation, several aspects had to be defined. This was done carefully using the data gathered from the previous reseach on \gls{ld}s. With this in mind, some tools were chosen to accelerate the development process, such as using of Unity and Figma.

Mockups were created and discussed with the team, going through several rounds of refinement. This is an extremely important step because it defines a big part of the user experience of the players. This was a great opportunity to step out of the confort zone and learn more about the experience of the players. After all, the experience has to be really well optimized for children with \gls{ld}s.

With designs in mind, the architecture was developed. The biggest challenge here was to create a system which allows for the creation of multiple games on top of it. For this we used the C\# language, an \gls{oop} language which supports the four fundamental principles of \gls{oop} (polymorphism, encapsulation, inheritance, and abstraction). With this and some design pricinples such as Observer and Singleton, we're able to write clean and reusable code. Because of this, the proposed solution includes a base that is the ''frame'' for all mini-games to be built on. This approach was not only used to save time and develop faster, but also to allow for future developments in the software. 

% TODO: Talk about the testing and analyzing phase

All these developments, led to the creation of a 2D game, containing multiple mini-games inside it. A game with a cartoonish look makes easy to understand instructions and visual cues, so that children with \gls{ld}s could experience the best of a learning game.

This thesis project allowed for a better understanding of the effectiveness of games as a tool to help young children learn, including those with specific challenges. Although we were only able to test with three out of the five mini-games, this really helped increasing the quality of the project. The testing phase engaged children and sparked interest, but also raised issues that needed to be accessed.

The task of developing games that target a niche of users allowed for an extense research on the matter. This crossover between multiple areas such as Psychology, \gls{ld}s, Software Engineering, and User Experience, leads to a product that can be useful. 

\section{Future work}
\label{futureWork}

Going forward, some things that would totally increase the quality of the games, would be the access for multiple languages. Right now, the games are tailored to a specific group of children and, the ability to add new levels is quite simple. This is due to the simplicity of the design of the software. Despite this, because these games are to be used in schools and medical centres, the end game is to let the teachers or doctors or anyone assisting the gameplay to children to be able to edit and create a completely different narrative for each children. What can be proposed is a new area to the main menu, where the tutor can manage each game and customize the gameplay. This could include changing the story, icons, and adding or removing levels.

