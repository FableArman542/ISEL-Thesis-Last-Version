%!TEX root = ../template.tex
%%%%%%%%%%%%%%%%%%%%%%%%%%%%%%%%%%%%%%%%%%%%%%%%%%%%%%%%%%%%%%%%%%%%
%% abstrac-en.tex
%% ISEL thesis document file
%%
%% Abstract in English
%%%%%%%%%%%%%%%%%%%%%%%%%%%%%%%%%%%%%%%%%%%%%%%%%%%%%%%%%%%%%%%%%%%%
% The dissertation must contain two versions of the abstract, one in the same language as the main text, another in a different language.  The package assumes that the two languages under consideration are always Portuguese and English.

% The package will sort the abstracts in the appropriate order. This means that the first abstract will be in the same language as the main text, followed by the abstract in the other language, and then followed by the main text. For example, if the dissertation is written in Portuguese, first will come the summary in Portuguese and then in English, followed by the main text in Portuguese. If the dissertation is written in English, first will come the summary in English and then in Portuguese, followed by the main text in English.

% The abstract shoul not exceed one page and should answer the following questions:

% The format of your abstract will depend on the discipline in which you are working. However, all abstracts generally cover the following five sections:
% \begin{itemize}
% 	\item The purpose of the research (what’s it about and why’s that important)
% 	\item The methodology (how you carried out the research)
% 	\item The key research findings (what answers you found)
% 	\item The implications of these findings (what these answers mean)
% \end{itemize}

The purpose of this thesis is the development and evaluation of five educational mini-games designed for children with learning disabilities. The main objective is to create a fun and accessible way to support children overcoming different learning disabilities.

Firstly, the development process approaches the theory behind the subject, performing a deep dive to find out what makes a learning disability. This is followed by a complete user profile that aims to detail each learning disability. This research is crucial for success in the creation process.

The implementation comes next. Gameplay requirements for the games are drawn and refined, allowing for the designing of the mockups of each game. Each game will then go through rounds of discussions to better expand the user experience of the children.

The games are then tested with children and results are gathered by both observing and interviewing them. The feedback can then be used as an input for some more implementations. Doing this in a cycle surely creates an even better experience and expanding of the product, overcoming their learning disabilities.

The key findings showed that the children are generally excited about playing the games. However, issues like complicated controls and unclear instructions make things harder for some of them. Despite these problems, the positive feedback suggested that the games managed to keep the children interested and could be helpful for learning.


% Palavras-chave do resumo em Inglês
\begin{keywords}
	accessibility, accessible learning, child education, cognitive challenges, educational mini-games, game-based learning, gameplay design, iterative development, learning disabilities, user experience, user feedback
\end{keywords}
	
