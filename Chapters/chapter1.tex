% 
%  chapter1.tex
%  ThesisISEL
%  
%  Created by Matilde Pós-de-Mina Pato on 2012/10/09.
%
\chapter{Introduction}
\label{cha:introduction}
\section{Motivation and Context}
Learning disabilities affect millions of children in the whole world, making it hard for them to get to their full potential. Those disabilities can make the reading comprehension, information memorization, mathematic processes, and the ability to focus on class way harder.

However, technology and games can offer new opportunities to help children with learning disabilities to overcome their challenges. Interactive educational mini games can be projected to teach specific learning abilities in a charismatic and enthusiastic way, while at the same time the players can have a fun and feel accomplishment. 

\section{Thesis Goals}
The goal of this project is to explore the potential of educational games to children with learning disabilities. We address the development of mini games having in consideration effective pedagogical approaches, adapted to meet the individual needs of each child. The games will be designed to be accessible and intuitive, providing an environment of growth and motivation.

Lastly, this research hopes to contribute to innovative technological solutions which can improve the quality of life of children with learning disabilities, offering them new opportunities to learn, grow, and achieve their full potential. 

Developing a game for children with learning disabilities is a challenging but rewarding project, as it has a positive impact on their learning and development. The focus of this project is to create an engaging and interactive series of mini-games that are tailored to the specific needs of children with learning disabilities, helping them to develop cognitive, social, and motor skills.

Technology can play a crucial role in the development of such games, as it offers many features that can make learning more accessible and enjoyable for children with disabilities. For instance, technologies like touchscreens, motion sensors, and voice recognition can make the game more interactive, allowing children to use a range of different skills to play and learn.

Moreover, technology can also provide real-time learning feedback, meaning that the game can adjust its difficulty level based on the child's performance, providing targeted support for their individual learning needs. This can help to ensure that the child is always challenged but not overwhelmed, making the game both enjoyable and beneficial.

Overall, developing a game for children with learning disabilities using technology can be an exciting and innovative way to support their learning and development. The project has the potential to contribute to the field of assistive technology and special education and can offer valuable insights into how technology can be used to enhance learning and support the needs of children with disabilities.

% \section{Solution Outline}
% (Apresentar uma figura com o diagrama geral da solução, 
% 	evidenciando quais os vários jogos a desenvolver)
% 	(Indicar os aspetos informáticos: ambiente a usar e interface 
% 	a usar com as crianças)

\section{Document Organization}
% (Nesta secção resumir a organização da restante tese)
% 	(Chapter 2 addresses..... In Chapter 3, we have....)

This thesis guides you through the development of educational mini-games designed for children with learning disabilities. Here’s how the document is laid out:

\paragraph{\ref{cha:introduction}. Introduction} Starts by discussing the motivation and context behind this project. An outline of the goals will be observed aiming to achieve and provide an overview of the proposed solution.

\paragraph{\ref{cha:related_work}. Related Work} In this chapter, a definition of learning disabilities is created and each type will be explored in detail followed by a market research of existing work related to understand what’s already available and how this project can contribute.

\paragraph{\ref{cha:games_to_be_developed}. Games to be Developed} Here, all the five mini-games are detailed: Hearts, Maze, Sounds, Words, and Puzzle. For each game, we will discuss the specific skills they target and explain how they’re designed to assist children with different learning challenges.

\paragraph{\ref{cha:game_development}. Game Development} This chapter covers how the mini-games were developed. Design principles, methodologies and tools will be set each with its motives. Each mini-game's architecture will also be explored.

\paragraph{\ref{cha:results}. Results} Finally, the outcome of the project will be presented. This includes testing results, feedback gathered, and an analysis of how well the mini-games achieved their main goals.


\newpage

% This package is distributed under GPLv3 License. If you have questions or \todo{A marginpar note!} doubts concerning the guarantees, rights and duties of those who use packages under GPLv3 License, please read \url{http://www.gnu.org/licenses/gpl.html}.

% \todo[inline]{A a note in a line by itself.}

% Please note that 
% \begin{center}
% 	\textbf{\large this package and template are not official for ISEL/IPL}.
% \end{center}
