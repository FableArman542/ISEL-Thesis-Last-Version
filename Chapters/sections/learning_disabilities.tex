\section{Learning Disabilities}
\label{section:learningdisabilityexplanation}
\gls{ld} are a variation of neurological conditions that can affect the capacity of a person to process, understand and remember information or specific learning abilities, even when a person’s intelligence is normal and learning opportunities are suitable.

% TODO: Adicionar Referencias

These conditions can affect abilities such as reading, mathematics, reasoning, logic, visual and auditive comprehension, memory, and information organization. The common types of learning disabilities include dyslexia and dyscalculia.%, and ADHD (Attention deficit and hyperactive disorder).

The main causes of learning disabilities are not fully understood, but it’s believed that genealogical, neurobiological, and environmental factors can be a root cause. It’s important to highlight that learning disabilities aren’t caused by lack of effort, motivation, or intelligence. 

The diagnostic and treatment of learning disabilities can include a combination of educational interventions, behavioural therapies and medication, depending on the seriousness and the individual needs of each affected people. It is known that the practice of educational games can also overcome learning disabilities.

\newpage
\section{Overcoming Learning Disabilities}
\label{sec:overcomingLds}
	% (Apresentar as definições das dificuldades de aprendizagem)
	% (Colocar as referências para a bibliografia da área da psicologia)
	% (Explicar as razões pelas quais os jogos são adequados para combater
	% as dificuldades de aprendizagem)

Children with learning disabilities face unique challenges both in their educational and social life. The use of interactive technology, specifically games, has shown great promise in helping these kids to deal with situations by providing an engaging learning experience specifically tailored for them. In this section, the state-of-the-art about the series of mini-games for supporting children with learning disabilities is addressed.

\textbf{Game Based Learning} \acrshort{gbl} has emerged as an innovative approach to address the specific needs of children with \gls{ld}'s. This technique utilizes the power of games to define and support learning outcomes. This is a great way to help them learn due to the fact that games are already a big part in most of the children's playtime. The evolution of technology has provided lots of different ways people can play games and we often see children playing more in their laptop, tablet and even their smartphones. A \acrshort{gbl} environment achieves these learning outcomes through educational games that have elements such as engagement, rewards, and healthy competition, making people stay motivated while learning \cite{gblProsCons}.
% TODO: Adicionar referencias
% https://www.edutopia.org/topic/game-based-learning/
% https://www.prodigygame.com/main-en/blog/game-based-learning/
% https://files.eric.ed.gov/fulltext/EJ1090277.pdf


Some important points should be addressed when designing a product like this. First of all, understanding the nature of \acrshort{ld}'s is crucial. \gls{ld}'s encompass a range of conditions that include Dyslexia, Dyscalculia, \acrfull{lpd}, Non-Verbal Learning Disabilities, Dysgraphia, Auditory Processing Disorder, and Visual Motor Deficit.

% Explain each learning disabilities
In order to properly develop these mini-games we should have in mind the characteristics of these disabilities. This way, some research was done so that we can better understand each \acrshort{ld}. For each one, we created a ``User Profile`` that contains the most likely Problems, Goals, Misconceptions, Needs, and Psychographics.