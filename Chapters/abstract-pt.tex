%!TEX root = ../template.tex
%%%%%%%%%%%%%%%%%%%%%%%%%%%%%%%%%%%%%%%%%%%%%%%%%%%%%%%%%%%%%%%%%%%%
%% abstrac-pt.tex
%% ISEL thesis document file
%%
%% Abstract in Portuguese
%%%%%%%%%%%%%%%%%%%%%%%%%%%%%%%%%%%%%%%%%%%%%%%%%%%%%%%%%%%%%%%%%%%%
% Independentemente da língua em que está escrita a dissertação, é necessário um resumo na língua do texto principal e um resumo noutra língua.  Assume-se que as duas línguas em questão serão sempre o Português e o Inglês.

% O \emph{template} colocará automaticamente em primeiro lugar o resumo na língua do texto principal e depois o resumo na outra língua.  Por exemplo, se a dissertação está escrita em Português, primeiro aparecerá o resumo em Português, depois em Inglês, seguido do texto principal em Português. Se a dissertação está escrita em Inglês, primeiro aparecerá o resumo em Inglês, depois em Português, seguido do texto principal em Inglês.

% O resumo não deve exceder uma página e deve responder às seguintes questões:
% \begin{itemize}
% % What's the problem?
% 	\item Qual é o problema?
% % Why is it interesting?
% 	\item Porque é que ele é interessante?
% % What's the solution?
% 	\item Qual é a solução?
% % What follows from the solution?
% 	\item O que resulta (implicações) da solução?
% \end{itemize}

O objetivo deste projeto é o desenvolvimento e a avaliação de cinco mini-jogos educativos concebidos para crianças com dificuldades de aprendizagem. O principal objetivo é criar uma forma divertida e acessível de apoiar as crianças a ultrapassar diferentes dificuldades de aprendizagem.

Em primeiro lugar, o processo de desenvolvimento aborda a teoria subjacente ao tema, efetuando uma extensa pesquisa para descobrir o que constitui uma perturbação da aprendizagem. Segue-se um perfil de utilizador que visa detalhar cada uma. Esta investigação é crucial para o sucesso do processo de desenvolvimento.

A implementação vem a seguir. Os requisitos de jogabilidade para os jogos são desenhados e refinados, permitindo a conceção de mockups para cada jogo. Cada jogo irá depois ser submetido a rondas de discussão para melhor expandir a experiência de utilização das crianças.

Os jogos são então testados com crianças e os resultados são recolhidos através da observação e de entrevistas com as crianças. O feedback pode então ser utilizado como um contributo para mais algumas implementações. Fazer isto num ciclo cria certamente uma experiência ainda melhor e uma possível expansão no desenvolvimento do produto.

As principais conclusões revelam que as crianças estão geralmente entusiasmadas com os jogos. No entanto, questões como controlos complicados e instruções pouco claras tornam as coisas mais difíceis para algumas delas. Apesar destes problemas, as reações positivas sugerem que os jogos conseguiram manter as crianças interessadas e podem ser extremamente úteis, ultrapassando as suas dificuldades de aprendizagem.

% Palavras-chave do resumo em Português
\begin{keywords}
	acessibilidade, aprendizagem acessível, aprendizagem baseada em jogos, conceção de jogos, desafios cognitivos, desenvolvimento iterativo, dificuldades de aprendizagem, educação de crianças, experiência de utilizador, feedback de utilizador, mini-jogos educativos
\end{keywords}
	
% to add an extra black line
